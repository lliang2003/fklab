\documentclass{article}
\usepackage{CJKutf8}
\usepackage{amsmath}
\usepackage{algorithm}
\usepackage{algorithmic}

\title{算法设计与复杂性理论 作业二}
\author{10848194  樊锴}
\date{2008-11-14}

\begin{document}
\begin{CJK}{UTF8}{gbsn}

\maketitle
\section{}
本题与0/1背包问题类似。用S(k,y)表示只放前k个货柜,且占用长度不超过y时的最大收益,则递推方程与边界条件为:
\begin{align*}
& S(k,y)=\max\{S(k-1,y), S(k-1, y-L_k)+v_k\}	\\
& S(k,0)=0, 	\quad 0\le k\le n	\\
& S(0,y)=0, 	\quad 0\le y\le D	\\
& S(k,y)=-\infty, 	\quad y<0	
\end{align*}

\begin{algorithm}
\caption{Storage}
\begin{algorithmic}[1]
\STATE INIT $S(k,0) \gets 0,\quad0 \le k\le n$
\STATE INIT $S(0,y) \gets 0,\quad0 \le y\le D$
\FOR    {$k=1$ to $n$}
\FOR    {$y=1$ to $D$}
    \IF     {$y<L_k$ OR $S(k-1,y)>S(k-1,y-L_k)+v_k$}
    \STATE  {$S(k,y) \gets S(k-1,y)$}
    \ELSE
    \STATE  {$S(k,y) \gets S(k-1,y-L_k)+v_k$ }
    \ENDIF
\ENDFOR
\ENDFOR
\RETURN $S(n,D)$
\end{algorithmic}
\end{algorithm}

算法的时间复杂度和空间复杂度都为$\Theta(n*D)$。

\section{}
以T(k,y)表示只用前k种硬币时支付y的最小重量,则递推函数和边界条件为:
\begin{align*}
& T(k,y)=\min\{T(k-1,y),T(k,y-v_k)+w_k\} \\
& T(1,y)=y*w_k,\quad y>0    \\
& T(k,0)=0,\quad 0 \le k\le n
\end{align*}

\begin{algorithm}
\caption{Coins}
\begin{algorithmic}[1]
\STATE INIT $T(k,0) \gets 0,\quad0 \le k\le n$
\STATE INIT $T(1,y) \gets y*w_1,\quad0 \le y\le Y$
\FOR    {$k=2$ to $n$}
\FOR    {$y=1$ to $Y$}
    \IF     {$y<v_k$ OR $T(k-1,y)<T(k-1,y-v_k)+w_k$}
    \STATE  {$T(k,y) \gets T(k-1,j)$}
    \ELSE
    \STATE  {$T(k,y) \gets T(k,y-v_k)+w_k$ }
    \ENDIF
\ENDFOR
\ENDFOR
\end{algorithmic}
\end{algorithm}
算法时间和空间复杂度都为$\Theta(n*Y)$。

备忘录表为:
\begin{tabular}[t]{l|c|c|c|c|c|c|c|c|c|c|c|c}
\hline
k\&y & 1 & 2 & 3 & 4 & 5 & 6 & 7 & 8 & 9 & 10 & 11 & 12  \\
\hline
1   & 1  & 2 & 3 & 4 & 5 & 6 & 7 & 8 & 9 & 10 & 11 & 12  \\
\hline
2   & 1  & 2 & 3 & 2 & 3 & 4 & 5 & 4 & 5 & 6 & 7 & 6  \\
\hline
3   & 1  & 2 & 3 & 2 & 3 & 4 & 5 & 4 & 5 & 6 & 7 & 6  \\
\hline
4   & 1  & 2 & 3 & 2 & 3 & 4 & 5 & 4 & 5 & 6 & 7 & 6  \\
\hline
\end{tabular}

标记函数的递推公式为:
\begin{displaymath}
i(k,y)=\left\{
\begin{array}{ll}
i(k-1,y),   & y<v_k\ OR\ T(k-1,y) > T(k,y-v_k)+w_k \\
k,          & T(k-1,y) \le T(k,y-v_k)+w_k 
\end{array} \right.
\end{displaymath}

标记函数表为:
\begin{tabular}[t]{l|c|c|c|c|c|c|c|c|c|c|c|c}
\hline
k\&y & 1 & 2 & 3 & 4 & 5 & 6 & 7 & 8 & 9 & 10 & 11 & 12  \\
\hline
1   & 1 & 1 & 1 & 1 & 1 & 1 & 1 & 1 & 1 & 1 & 1 & 1 \\
\hline
2   & 1 & 1 & 1 & 2 & 1 & 1 & 1 & 2 & 1 & 1 & 1 & 2 \\
\hline
3   & 1 & 1 & 1 & 2 & 1 & 3 & 1 & 2 & 1 & 1 & 1 & 2 \\
\hline
4   & 1 & 1 & 1 & 2 & 1 & 3 & 1 & 2 & 1 & 1 & 1 & 2 \\
\hline
\end{tabular}

所以最后应付3枚第二种硬币,总重量为6。

\section{}
用T(k,y,d)表示只放前k个物品,重量不超过y且体积不超过d的最大价值,则递推函数和边界条件为:
\begin{align*}
& T(k,y,d)=\min\{T(k-1,y,d),T(k,y-w_k,d-c_k)+v_k\} \\
& T(0,y,d)=0, \quad y>0,d>0    \\
& T(k,y,0)=0, \quad 0 \le k\le n    \\
& T(k,0,d)=0, \quad 0 \le k\le n    \\
& T(k,y,d)=-\infty, \quad y<0\ OR\ d<0
\end{align*}

\begin{algorithm}
\caption{Package}
\begin{algorithmic}[1]
\STATE INIT $T(0,y,d) \gets 0,\quad0 \le y\le W,0\le d\le V$
\STATE INIT $T(k,0,0) \gets 0,\quad0 \le k\le n$
\FOR    {$k=1$ to $n$}
\FOR    {$y=1$ to $W$}
\FOR    {$d=1$ to $V$}
    \IF     {$y<w_k$ OR $d<c_k$ OR $T(k-1,y,d)>T(k-1,y-w_k,d-c_k)+v_k$}
    \STATE  {$T(k,y,d) \gets T(k-1,y,d)$}
    \ELSE
    \STATE  {$T(k,y,d) \gets T(k,y-w_k,d-c_k)+v_k$ }
    \ENDIF
\ENDFOR
\ENDFOR
\ENDFOR
\end{algorithmic}
\end{algorithm}

算法时间复杂度和空间复杂度都为$\Theta(n*W*V)$。

\section{}
每次合并的比较次数为两个数组的规模之和。
用D(i,j)表示从i到j的数组的总规模,T(i,j)来表示合并从i到j的数组所需的代价,i可以大于j。
\begin{align*}
& T(i,j)=\min_{i\le k\le j}{\{T(i,k)+T(k,j)+D(i,j)\}} \\
& T(i,i)=D(i,i), \quad0\le i <n
\end{align*}
使用$\Theta(n)$的预处理后,D(i,j)可以用$\Theta(1)$时间求解,方法为:
记S(i)表示从0到i的数组的总规模,显然可在$\Theta(n)$的时间求出,
则对任意的i,j,
\[  D(i,j)=\left\{
\begin{array}{ll}
S[j],   & i=0 \\
S[j]-S[i-1], & i>0\ AND\ i\le j \\
S[n-1]-S[i-1]+S[j], & j\ge0\ AND\ i>j
\end{array} \right.
\]
因此合并算法的时间复杂度为$\Theta(n^3)$。

本题也可以用贪心法求解,即每次选择规模之和最小的相邻数组合并,直至只剩一个数组即可。
这样算法时间复杂度为$\Theta(n^2)$,若采用堆来保存相邻数组的规模和,可以进一步把时间复杂度减至$\Theta(nlogn)$。

\section{}
用$T_k$表示以第k个元素结尾的最长子序列的长度,$I$为标记函数。
\begin{align*}
& T_k=\max_{0<j<k\land A[k]>A[j]}{T_j+1} \\
& I_k=\left\{
\begin{array}{ll}
arg\max_{0<j<k\land A[k]>A[j]}{T_j},   & T_k>1 \\
k,                                      & T_k=1 
\end{array} \right.
\end{align*}

\begin{algorithm}
\caption{LIS}
\begin{algorithmic}[1]
\STATE INIT $T(k) \gets 1,\quad 0<k\le n$
\STATE INIT $I(k) \gets k,\quad 0<k\le n$
\STATE $maxlen \gets 1$
\FOR    {$i=2$ TO $n$}
    \FOR    {$j=1$ TO $i-1$}
        \IF     {$A[i]>A[j]$ AND $T[i]<T[j]+1$}
        \STATE  {$T[i] \gets T[j]+1, I[i] \gets j$}
        \ENDIF
    \ENDFOR
    \IF     {$T[i]>maxlen$}
    \STATE  {$maxlen \gets T[i]$}
    \ENDIF
\ENDFOR
\FOR    {$i=1$ TO $n$}
    \IF     {$T[i]=maxlen$}
    \STATE  OUTPUTLIS(I, i)
    \STATE  print(';')
    \ENDIF
\ENDFOR
\end{algorithmic}
\end{algorithm}

\begin{algorithm}
\caption{OUTPUTLIS(I, i)}
\begin{algorithmic}[1]
\IF     {$I[i] \ne i$}
\STATE  OUTPUTLIS(I, I(i))
\ENDIF
\STATE  print(i)
\end{algorithmic}
\end{algorithm}

算法的时间复杂度为$\Theta(n^2)$,空间复杂度为$\Theta(n)$。

\section{}
满足多米诺性质,可以用回溯法求解,采用深度优先搜索。
用$T_k$表示第k种配件的选择,解向量为$<T_1,T_2,T_3,T_4>$。
约束条件为
\[V_1+V_2+V_3+V_4 \le 120  \]
记$B_k$为第k种配件中最轻的,则对于一个部分解$<T_1,\dots,T_k>$,代价函数为
\[  \sum_{i=1}^{k}{V[T_k]}+\sum_{i=k+1}^{4}{V[B_k]} \]
一旦在搜索中不满足约束条件或者代价函数大于已知最优解,即可回溯。

本题最小重量为31,四种配件分别选择第3,1,2,3个供应商,总价值为119。

\section{}
把问题拓展为求$\sum_{0<k\le n}{A_k*x_k}\le S$的所有非负整数解,
其解向量为$<x_1,x_2,\dots,x_n>$,对于$x_k$的可行范围为$[0,S/A_k]$。

\begin{algorithm}
\caption{CALC(A[n],S)}
\begin{algorithmic}[1]
\STATE INIT $X_k \gets 0, \quad 0<k\le n$
\STATE $k \gets 1$
\WHILE {$k>1$}
    \IF     {$k=n\ AND\ S\ge 0$}
        \STATE  OUTPUT X
    \ENDIF
    \IF {$S<0\ OR\ k>n$}
        \STATE  $S \gets S+A_k*X_k$
        \STATE  $X_k \gets 0$
        \STATE  $k  \gets k-1$
        \STATE  $X_k \gets X_k+1$
        \STATE  $S \gets S-A_k*X_k$
    \ELSE 
        \STATE  $k \gets k+1$
    \ENDIF
\ENDWHILE
\end{algorithmic}
\end{algorithm}
对于此题中的实例,解为
0 0 0;
0 0 1;
0 0 2;
0 0 3;
0 0 4;
0 0 5;
0 0 6;
0 1 0;
0 1 1;
0 1 2;
0 1 3;
0 1 4;
0 2 0;
0 2 1;
0 2 2;
0 3 0;
1 0 0;
1 0 1;
1 0 2;
1 0 3;
1 0 4;
1 1 0;
1 1 1;
1 1 2;
1 2 0;
2 0 0;
2 0 1;
2 0 2;
2 0 3;
2 1 0;
2 1 1;
3 0 0;
3 0 1;
4 0 0;

\end{CJK}
\end{document}
